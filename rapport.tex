\documentclass[12pt]{article}
\usepackage[utf8]{inputenc}
\usepackage[T1]{fontenc}
\usepackage[french]{babel}
\usepackage{amsmath,amsfonts,amssymb}
\usepackage{fullpage}
\usepackage{graphicx}
\usepackage{hyperref}

\title{Virtualisation de performances}
\author{Aurèle Barrière \& Benjamin Fasquelle}

\begin{document}
\maketitle
\def\code#1{\texttt{#1}}

\section{Préambule}
Après avoir mis une adresse IP valide, il est bien possible de se connecter à la machine virtuelle avec \code{virsh console wheezy0} et installer \code{htop} (avec \code{apt-get} par exemple).

\section{Performances CPU}

\paragraph{Question 2.1.} Nous avons lancé le programme \texttt{hanoi} avec 15,17 et 18 comme arguments sur la machine virtuelle (créée avec \code{virsh}) et sur la Debian hôte. Nous sommes ainsi garantis que le programme tourne bien sur le même matériel physique.

Le temps d'exécution, en secondes est le suivant:

\begin{tabular}{| l | c | c | c |}\hline
Machine & $N=15$ & $N=17$ & $N=18$ \\\hline
Machine virtuelle & 10.169 & 40.975 & 181.521 \\\hline
Debian hôte & 0.132 & 0.408 & 0.840\\\hline
\end{tabular}

Il y a donc une différence de performance significative. Pour ces arguments, la machine virtuelle est entre 77 et 216 fois plus lente que sa machine hôte. Cela provient de la couche de virtualisation.

\paragraph{Question 2.2.} 
Il faut refaire les mesures.


----- NOTES

Je refais sur une machine puis sur 2, mais sur d'autres nodes qu'avant. il faut donc préciser qu'il ne faut pas comparer avec la question précédente.

hanoi(20)

sur une seule VM: 2min45

sur deux VM du meme serveur, sans rien forcer: 2m50 et 2m51





-----




Le temps d'exécution est du même ordre qu'avec une seule machine virtuelle. En utilisant \code{htop}, nous avons observer l'allocation des ressources CPU aux deux machines virtuelles exécutant \texttt{hanoi}. Nous avons pu voir que le CPU associé à l'une des machines virtuelles est dynamique: il était différent à chaque exécution, et changeait même pendant l'exécution. De ce fait, les exécutions parallèles d'\texttt{hanoi} ne s'exécutaient pas sur le même CPU, ce qui explique que le temps d'exécution (en parallèle) n'a pas presque pas augmenté par rapport à une exécution simple.


\paragraph{Question 2.3.}
On utilise la commande \code{virsh vcpupin wheezy0 0 0} pour mettre les deux machines virtuelles sur le même CPU.


Il faut refaire les mesures.

\begin{tabular}{| l | c |}\hline
Utilisation des CPU & $N=25$ \\\hline
Sans forcer le CPU & 1.532 \\\hline
En forçant le CPU & 1.580 \\\hline
\end{tabular}

ATTENTION: MESURES PAS LOGIQUES, TROP FAIBLES

-----NOTES

Bon la c'est bizarre. j'ai utilisée la meme commande que ce qu'on a marqué.

C'est bien plus lent : 7m30 et 7m31.

Mais avec htop, je voyais 4 barres actives (pas 1). Alors c'est mieux qu'avant, c'etait toujours les 4 memes (alors qu'avant ca se baladait sur les 40) mais ca devrait pas dépasser comme ça.

Sinon, le résultat reste concluant


-----


Cette fois, les deux programmes s'exécutant sur le même CPU, le temps d'exécution est augmenté: il est environ le double que pour une exécution simple. A CHANGER

\section{Performances réseau}
\paragraph{Question 3.1.}
On note les résultats de la commande \code{ping}. Dans tous les cas testés, il n'y a eu aucune perte de paquets. On ne notera donc que la dernière ligne de \code{ping}.
\\

Entre deux VMs hébergées sur le même serveur:

\code{rtt min/avg/max/mdev = 0.235/0.286/0.378/0.047 ms}
\\

Entre deux VMs hébergées sur des serveur différents:

\code{rtt min/avg/max/mdev = 0.354/0.444/0.819/0.145 ms}
\\

Entre deux serveurs différents:

\code{rtt min/avg/max/mdev = 0.088/0.103/0.190/0.034 ms}
\\

Entre un OS hôte et une VM sur le même serveur:

\code{rtt min/avg/max/mdev = 0.240/0.296/0.419/0.051 ms}
\\

On remarque que la communication entre serveurs est la plus rapide, tandis que celles avec les machines virtuelles sont les plus lentes. On peut en déduire que les machines virtuelles ralentissent les communications.

On peut également observer que les discussions entre les deux machines virtuelles sont plus rapides lorqu'elles sont hébergées sur le même serveur, on peut donc en déduire qu'il y a une communication interne au serveur, plus rapide que celle entre serveurs.



A REVOIR

JE NE SUIS PAS SUR DE COMMENT TOUT EXPLIQUER

\paragraph{Question 3.2.}
On donne les résultats de la commande \code{iperf}.
\\

Pour des machines hébergées sur le même serveur:

\code{[  3]  0.0-10.0 sec  2.76 GBytes  2.37 Gbits/sec}
\\

Pour des machines hébergées sur des serveurs différents:

\code{[  3]  0.0-10.0 sec  1.10 GBytes   941 Mbits/sec}
\\

Il y a donc une différence significative de bande passante (avec un facteur d'environ 2.5). Sans surpise, des machines hébergées sur le même serveur ont une bande passante bien supérieure à deux machines hébergées sur des serveurs différents.

ON N A PAS SATURE LE RESEAU

En émettant du trafic avec toutes les machines virtuelles, on parvient facilement à saturer le réseau.

\section{Migrons}
\paragraph{Question 4.1.}

Avec la commande \code{virsh migrate}, nous avons migrer une machine virtuelle d'un serveur à un autre. Cette opération est rapide (temps réel: 5 secondes, temps système: 8ms).

\paragraph{Question 4.2.}

On peut migrer deux machines virtuelles en parrallèle d'un même serveur source vers un même serveur destination. Comme attendu, l'opération est plus longue que lorsqu'on migre une seule machine virtuelle (temps réel: 8secondes, temps système: 40ms).


On peut également migrer en parallèle deux machines virtuelles depuis des serveurs sources différents. Ici, l'opération est environ deux fois plus longue que lorsqu'on migre une seule machine virtuelle (temps réel: 3secondes, temps système: 16ms).



\paragraph{Question 4.3.}

On peut migrer en parallèle une machine virtuelle d'un serveur A vers un serveur B, et une autre du serveur B vers le serveur A. Comme attendu, le temps de migration est deux fois plus élevé que pour une simple migration.

\paragraph{Question 4.4.}

Le temps de migration est plus long?

\paragraph{Question 4.5.}

La durée de migration est similaire?

\paragraph{Question 4.6.}

A FAIRE


\end{document}

2.1 
dans le noeud (parapluie-21), pour hanoi avec n=15
real	0m0.305s
user	0m0.008s
sys	0m0.132s

n= 17
real	0m0.991s
user	0m0.072s
sys	0m0.408s

n=18
real	0m1.982s
user	0m0.144s
sys	0m0.840s


sur la machine virtuelle (virsh) :
n=15
real	0m10.286s
user	0m0.116s
sys	0m10.169s

n=17
real	0m41.431s
user	0m0.452s
sys	0m40.975s


n=18
real	1m22.350s
user	0m0.820s
sys	1m21.521s


2.2
hanoi 25, sans les forcer sur un processeur
real	0m14.416s
user	0m11.733s
sys	0m1.508s

real	0m14.724s
user	0m11.989s
sys	0m1.532s

hanoi 25, en les forçant sur un même processeur (virsh vcpupin wheezy0 0 0)
real	0m24.501s
user	0m22.573s
sys	0m1.360s

real	0m24.316s
user	0m22.009s
sys	0m1.580s


3.1
deux machines virtuelles sur le meme serveur :
64 bytes from 10.158.3.253: icmp_req=1 ttl=63 time=0.378 ms
64 bytes from 10.158.3.253: icmp_req=2 ttl=63 time=0.280 ms
64 bytes from 10.158.3.253: icmp_req=3 ttl=63 time=0.300 ms
64 bytes from 10.158.3.253: icmp_req=4 ttl=63 time=0.246 ms
64 bytes from 10.158.3.253: icmp_req=5 ttl=63 time=0.288 ms
64 bytes from 10.158.3.253: icmp_req=6 ttl=63 time=0.280 ms
64 bytes from 10.158.3.253: icmp_req=7 ttl=63 time=0.235 ms
--- 10.158.3.253 ping statistics ---
7 packets transmitted, 7 received, 0% packet loss, time 5997ms
rtt min/avg/max/mdev = 0.235/0.286/0.378/0.047 ms

deux machines virtuelles sur des serveurs différents
PING 10.158.3.253 (10.158.3.253) 56(84) bytes of data.
64 bytes from 10.158.3.253: icmp_req=1 ttl=64 time=0.819 ms
64 bytes from 10.158.3.253: icmp_req=2 ttl=64 time=0.386 ms
64 bytes from 10.158.3.253: icmp_req=3 ttl=64 time=0.430 ms
64 bytes from 10.158.3.253: icmp_req=4 ttl=64 time=0.407 ms
64 bytes from 10.158.3.253: icmp_req=5 ttl=64 time=0.384 ms
64 bytes from 10.158.3.253: icmp_req=6 ttl=64 time=0.354 ms
64 bytes from 10.158.3.253: icmp_req=7 ttl=64 time=0.374 ms
64 bytes from 10.158.3.253: icmp_req=8 ttl=64 time=0.405 ms
^C
--- 10.158.3.253 ping statistics ---
8 packets transmitted, 8 received, 0% packet loss, time 7011ms
rtt min/avg/max/mdev = 0.354/0.444/0.819/0.145 ms

entre les deux parapluies
PING parapluie-21.rennes.grid5000.fr (172.16.99.21) 56(84) bytes of data.
64 bytes from parapluie-21.rennes.grid5000.fr (172.16.99.21): icmp_req=1 ttl=64 time=0.190 ms
64 bytes from parapluie-21.rennes.grid5000.fr (172.16.99.21): icmp_req=2 ttl=64 time=0.088 ms
64 bytes from parapluie-21.rennes.grid5000.fr (172.16.99.21): icmp_req=3 ttl=64 time=0.093 ms
64 bytes from parapluie-21.rennes.grid5000.fr (172.16.99.21): icmp_req=4 ttl=64 time=0.094 ms
64 bytes from parapluie-21.rennes.grid5000.fr (172.16.99.21): icmp_req=5 ttl=64 time=0.090 ms
64 bytes from parapluie-21.rennes.grid5000.fr (172.16.99.21): icmp_req=6 ttl=64 time=0.093 ms
64 bytes from parapluie-21.rennes.grid5000.fr (172.16.99.21): icmp_req=7 ttl=64 time=0.090 ms
64 bytes from parapluie-21.rennes.grid5000.fr (172.16.99.21): icmp_req=8 ttl=64 time=0.090 ms
^C
--- parapluie-21.rennes.grid5000.fr ping statistics ---
8 packets transmitted, 8 received, 0% packet loss, time 6998ms
rtt min/avg/max/mdev = 0.088/0.103/0.190/0.034 ms


depuis l'os hote(parapluie) a une VM du meme serveur
PING 10.158.3.253 (10.158.3.253) 56(84) bytes of data.
64 bytes from 10.158.3.253: icmp_req=1 ttl=63 time=0.419 ms
64 bytes from 10.158.3.253: icmp_req=2 ttl=63 time=0.304 ms
64 bytes from 10.158.3.253: icmp_req=3 ttl=63 time=0.289 ms
64 bytes from 10.158.3.253: icmp_req=4 ttl=63 time=0.247 ms
64 bytes from 10.158.3.253: icmp_req=5 ttl=63 time=0.296 ms
64 bytes from 10.158.3.253: icmp_req=6 ttl=63 time=0.284 ms
64 bytes from 10.158.3.253: icmp_req=7 ttl=63 time=0.240 ms
64 bytes from 10.158.3.253: icmp_req=8 ttl=63 time=0.289 ms
^C
--- 10.158.3.253 ping statistics ---
8 packets transmitted, 8 received, 0% packet loss, time 6999ms
rtt min/avg/max/mdev = 0.240/0.296/0.419/0.051 ms


3.2
iperf entre deux vms sur le meme serveur
iperf -s
iperf -c (adresse du serveur)
root@virtual-158-3-253:~# iperf -c 10.158.3.254
------------------------------------------------------------
Client connecting to 10.158.3.254, TCP port 5001
TCP window size: 23.5 KByte (default)
------------------------------------------------------------
[  3] local 10.158.3.253 port 58930 connected with 10.158.3.254 port 5001
[ ID] Interval       Transfer     Bandwidth
[  3]  0.0-10.0 sec  2.76 GBytes  2.37 Gbits/sec

iperf entre deux vms sur des serveurs différents
root@virtual-158-3-253:~# iperf -c 10.158.3.252
------------------------------------------------------------
Client connecting to 10.158.3.252, TCP port 5001
TCP window size: 23.5 KByte (default)
------------------------------------------------------------
[  3] local 10.158.3.253 port 47499 connected with 10.158.3.252 port 5001
[ ID] Interval       Transfer     Bandwidth
[  3]  0.0-10.0 sec  1.10 GBytes   941 Mbits/sec



4.1
time virsh migrate wheezy1 qemu+ssh://parapluie-32/system


real	0m5.840s
user	0m0.040s
sys	0m0.008s


4.2
time $(virsh migrate wheezy1 qemu+ssh://parapluie-32/system & virsh migrate wheezy3 qemu+ssh://parapluie-32/system)

real	0m7.850s
user	0m0.060s
sys	0m0.040s

$
envoi des deux vm depuis des serveurs sources différents
time $(virsh migrate wheezy3 qemu+ssh://parapluie-21/system)

real	0m2.820s
user	0m0.036s
sys	0m0.016s


time $(virsh migrate wheezy1 qemu+ssh://parapluie-32/system)

real	0m6.061s
user	0m0.048s
sys	0m0.008s


4.4
